% Options for packages loaded elsewhere
\PassOptionsToPackage{unicode}{hyperref}
\PassOptionsToPackage{hyphens}{url}
\PassOptionsToPackage{dvipsnames,svgnames,x11names}{xcolor}
%
\documentclass[
  letterpaper,
  DIV=11,
  numbers=noendperiod]{scrartcl}

\usepackage{amsmath,amssymb}
\usepackage{iftex}
\ifPDFTeX
  \usepackage[T1]{fontenc}
  \usepackage[utf8]{inputenc}
  \usepackage{textcomp} % provide euro and other symbols
\else % if luatex or xetex
  \usepackage{unicode-math}
  \defaultfontfeatures{Scale=MatchLowercase}
  \defaultfontfeatures[\rmfamily]{Ligatures=TeX,Scale=1}
\fi
\usepackage{lmodern}
\ifPDFTeX\else  
    % xetex/luatex font selection
\fi
% Use upquote if available, for straight quotes in verbatim environments
\IfFileExists{upquote.sty}{\usepackage{upquote}}{}
\IfFileExists{microtype.sty}{% use microtype if available
  \usepackage[]{microtype}
  \UseMicrotypeSet[protrusion]{basicmath} % disable protrusion for tt fonts
}{}
\makeatletter
\@ifundefined{KOMAClassName}{% if non-KOMA class
  \IfFileExists{parskip.sty}{%
    \usepackage{parskip}
  }{% else
    \setlength{\parindent}{0pt}
    \setlength{\parskip}{6pt plus 2pt minus 1pt}}
}{% if KOMA class
  \KOMAoptions{parskip=half}}
\makeatother
\usepackage{xcolor}
\setlength{\emergencystretch}{3em} % prevent overfull lines
\setcounter{secnumdepth}{-\maxdimen} % remove section numbering
% Make \paragraph and \subparagraph free-standing
\ifx\paragraph\undefined\else
  \let\oldparagraph\paragraph
  \renewcommand{\paragraph}[1]{\oldparagraph{#1}\mbox{}}
\fi
\ifx\subparagraph\undefined\else
  \let\oldsubparagraph\subparagraph
  \renewcommand{\subparagraph}[1]{\oldsubparagraph{#1}\mbox{}}
\fi


\providecommand{\tightlist}{%
  \setlength{\itemsep}{0pt}\setlength{\parskip}{0pt}}\usepackage{longtable,booktabs,array}
\usepackage{calc} % for calculating minipage widths
% Correct order of tables after \paragraph or \subparagraph
\usepackage{etoolbox}
\makeatletter
\patchcmd\longtable{\par}{\if@noskipsec\mbox{}\fi\par}{}{}
\makeatother
% Allow footnotes in longtable head/foot
\IfFileExists{footnotehyper.sty}{\usepackage{footnotehyper}}{\usepackage{footnote}}
\makesavenoteenv{longtable}
\usepackage{graphicx}
\makeatletter
\def\maxwidth{\ifdim\Gin@nat@width>\linewidth\linewidth\else\Gin@nat@width\fi}
\def\maxheight{\ifdim\Gin@nat@height>\textheight\textheight\else\Gin@nat@height\fi}
\makeatother
% Scale images if necessary, so that they will not overflow the page
% margins by default, and it is still possible to overwrite the defaults
% using explicit options in \includegraphics[width, height, ...]{}
\setkeys{Gin}{width=\maxwidth,height=\maxheight,keepaspectratio}
% Set default figure placement to htbp
\makeatletter
\def\fps@figure{htbp}
\makeatother
% definitions for citeproc citations
\NewDocumentCommand\citeproctext{}{}
\NewDocumentCommand\citeproc{mm}{%
  \begingroup\def\citeproctext{#2}\cite{#1}\endgroup}
\makeatletter
 % allow citations to break across lines
 \let\@cite@ofmt\@firstofone
 % avoid brackets around text for \cite:
 \def\@biblabel#1{}
 \def\@cite#1#2{{#1\if@tempswa , #2\fi}}
\makeatother
\newlength{\cslhangindent}
\setlength{\cslhangindent}{1.5em}
\newlength{\csllabelwidth}
\setlength{\csllabelwidth}{3em}
\newenvironment{CSLReferences}[2] % #1 hanging-indent, #2 entry-spacing
 {\begin{list}{}{%
  \setlength{\itemindent}{0pt}
  \setlength{\leftmargin}{0pt}
  \setlength{\parsep}{0pt}
  % turn on hanging indent if param 1 is 1
  \ifodd #1
   \setlength{\leftmargin}{\cslhangindent}
   \setlength{\itemindent}{-1\cslhangindent}
  \fi
  % set entry spacing
  \setlength{\itemsep}{#2\baselineskip}}}
 {\end{list}}
\usepackage{calc}
\newcommand{\CSLBlock}[1]{\hfill\break\parbox[t]{\linewidth}{\strut\ignorespaces#1\strut}}
\newcommand{\CSLLeftMargin}[1]{\parbox[t]{\csllabelwidth}{\strut#1\strut}}
\newcommand{\CSLRightInline}[1]{\parbox[t]{\linewidth - \csllabelwidth}{\strut#1\strut}}
\newcommand{\CSLIndent}[1]{\hspace{\cslhangindent}#1}

\KOMAoption{captions}{tableheading}
\makeatletter
\@ifpackageloaded{caption}{}{\usepackage{caption}}
\AtBeginDocument{%
\ifdefined\contentsname
  \renewcommand*\contentsname{Table of contents}
\else
  \newcommand\contentsname{Table of contents}
\fi
\ifdefined\listfigurename
  \renewcommand*\listfigurename{List of Figures}
\else
  \newcommand\listfigurename{List of Figures}
\fi
\ifdefined\listtablename
  \renewcommand*\listtablename{List of Tables}
\else
  \newcommand\listtablename{List of Tables}
\fi
\ifdefined\figurename
  \renewcommand*\figurename{Figure}
\else
  \newcommand\figurename{Figure}
\fi
\ifdefined\tablename
  \renewcommand*\tablename{Table}
\else
  \newcommand\tablename{Table}
\fi
}
\@ifpackageloaded{float}{}{\usepackage{float}}
\floatstyle{ruled}
\@ifundefined{c@chapter}{\newfloat{codelisting}{h}{lop}}{\newfloat{codelisting}{h}{lop}[chapter]}
\floatname{codelisting}{Listing}
\newcommand*\listoflistings{\listof{codelisting}{List of Listings}}
\makeatother
\makeatletter
\makeatother
\makeatletter
\@ifpackageloaded{caption}{}{\usepackage{caption}}
\@ifpackageloaded{subcaption}{}{\usepackage{subcaption}}
\makeatother
\ifLuaTeX
  \usepackage{selnolig}  % disable illegal ligatures
\fi
\usepackage{bookmark}

\IfFileExists{xurl.sty}{\usepackage{xurl}}{} % add URL line breaks if available
\urlstyle{same} % disable monospaced font for URLs
\hypersetup{
  pdfauthor={Equipo EDUMER},
  colorlinks=true,
  linkcolor={blue},
  filecolor={Maroon},
  citecolor={Blue},
  urlcolor={Blue},
  pdfcreator={LaTeX via pandoc}}

\author{}
\date{}

\begin{document}

\section{The socialization of meritocracy and market justice preferences
at
school}\label{the-socialization-of-meritocracy-and-market-justice-preferences-at-school}

\subsection{Introduction}\label{introduction}

Since its origins, educational institutions have been related to the
idea of social mobility and access to better opportunities. Therefore,
the consistent evidence of the high level of social reproduction at the
school level represents a threat to the promise of education and a
meritocratic system (\citeproc{ref-bourdieu_reproduction_1990}{Bourdieu
and Passeron 1990}). A large part of the research in this field at an
international level has addressed the extent to which the social origin
of students affects their academic results and their life opportunities
(\citeproc{ref-vonhippel_test_2019}{Von Hippel and Hamrock 2019}),
confirming that schools have severe difficulties in closing the gaps of
origin. Besides this socioeconomic perspective on school opportunities,
recent research has addressed to what extent inequalities in the school
context are also influencing students' perceptions, beliefs, and
attitudes: Are social inequalities perceived at the school context? Are
they rejected by the students, particularly those who are worst-off in
socioeconomic terms? Or, Is there evidence at the school level that
social inequalities are tolerated and even justified?
(\citeproc{ref-batruch_belief_2022}{Batruch et al. 2022};
\citeproc{ref-wiederkehr_belief_2015}{Wiederkehr et al. 2015}).~

In the present paper, we deal with the justification of social
inequalities by eighth-grade students in Chile, a country characterized
by a highly stratified educational system. In particular, we focus on
market justice preferences (Lane, 1986), which refer to the preferences
for distributing public goods (as health and education) based on
criteria such as competition and payment capacity. Although from a
rational point of view it could be expected an opposition to market
justice by the underprivileged majority,~ we argue that in a social
environment characterized by a promotion of meritocratic ideals - as the
school system - would lead to the opposite: a larger market justice
preferences.~

Given that the school environment has an important focus on performance,
achievement and acknowledgment, meritocracy has been one of the
principal concepts used for understanding and even for justifying
performance differences among students. Meritocracy is a distributive
system based on the belief that people should be rewarded and promoted
based on their abilities, knowledge, and achievements
(\citeproc{ref-young_rise_1958}{Young 1958}). It is often seen as a way
to create equal opportunities and fairness, as individuals can rise to
positions of power and influence based on their own merit rather than
their background or connections. However, some argue that meritocracy
can actually lead to tolerating or even justifying social inequalities,
as it can create a hierarchy where those who already have resources and
advantages are more likely to succeed. In this regard, a great deal of
academic research about meritocracy delves into the assessment of to
what extent rewards and privileges in society are related to merit,
emphasizing the so-called unfulfillable promise of meritocracy
(\citeproc{ref-mijs_stratified_2016}{J. Mijs 2016}). Complementing this
agenda, a second and emerging research area deals with subjective
aspects of meritocracy, such as perceptions and beliefs.~

{[}definición de market justice, welfare marketization, (Lindh){]}

{[}contexto Chileno market justice, dar cifras/porcentajes de
privatización de salud, educación y pensiones; asociar al tema de la
calidad (lo público es peor){]}

The perception of meritocracy refers to how individuals view and
understand the concept of meritocracy in their own society
(\citeproc{ref-duru-bellat_who_2012}{Duru-Bellat and Tenret 2012};
\citeproc{ref-castillo_meritocracia_2019}{Juan C. Castillo et al.
2019}). This perception can vary greatly depending on individual
experiences, social, economic, and cultural background. Some people may
see meritocracy as a fair and just system that allows anyone to succeed
based on their abilities and hard work. In contrast, others may view it
as a myth or a cover for existing power dynamics and inequality, serving
to maintain and even reinforce inequality
(\citeproc{ref-lampert_meritocratic_2013}{Lampert 2013};
\citeproc{ref-mijs_paradox_2021}{J. J. B. Mijs 2021}). Based in this
last perspective, we argue that individuals with a higher perception of
meritocracy will show a larger justification of social inequalities, as
individual achievement would be seen as rewarded and social policies as
less necessary (\citeproc{ref-batruch_belief_2022}{Batruch et al.
2022}).

Most of the research that has related meritocratic beliefs to inequality
justification so far has only considered adults, leaving aside the study
of how beliefs in this field develop at student age as well as the
impact of the school context and the family as the main socialization
agencies. Regarding schools, the way in which they deal with unequal
conditions of origin has been linked to the \emph{hidden curriculum}
(\citeproc{ref-chafel_schooling_1997}{Chafel 1997}), whereby students
learn about distributive norms in society and mechanisms of
justification of social differences. Based on recent studies that relate
school meritocracy to the justification of economic inequalities in the
adult population (\citeproc{ref-batruch_belief_2022}{Batruch et al.
2022}; \citeproc{ref-wiederkehr_belief_2015}{Wiederkehr et al. 2015}),
the central hypothesis guiding this research is that school-age students
with a higher perception of meritocracy - both at school and at the
societal level - will show a larger justification of social
inequalities, as individual achievement would be seen as appropriately
rewarded and social mechanisms for correcting inequalities as less
necessary (\citeproc{ref-batruch_belief_2022}{Batruch et al. 2022}).

\textasciitilde{}

{[}introducir distinción sobre percepción de meritocracia a nivel
escolar y a nivel de la sociedad{]}

{[}resaltar 3 focos de innovación: market justice preferences at school
level, relación con meritocracia, y esquema de meritocracia en la
escuela y en la sociedad{]}

The present paper deals with the association between the perception of
meritocracy and the justification of social inequalities, with two main
focuses. Firstly, it assesses the justification of social inequalities
in social policy domains, such as health, pensions, and education. We
argue that individuals who perceive more meritocracy would be more
willing to justify better services in these domains for those with
higher incomes.~ Secondly, we focus on the student-age population as we
point out that it is possible to track down the origin of meritocratic
beliefs (and their consequences) to early socialization processes. To
this regard, we take into account the family and the school as two main
socialization agencies that play a significant role in the socialization
of cultural beliefs by transmitting cultural norms, values, and
expectations to young people.~

\subsection{Justification of
inequality}\label{justification-of-inequality}

Research on social stratification beliefs, which explore individual
perceptions of who deserves what and why
(\citeproc{ref-kluegel_beliefs_1987}{Kluegel and Smith 1987}),
highlights that people's explanations and justifications of social
inequality are closely tied to their judgments of deservingness. The
influence of ideologies (\citeproc{ref-wegener_dominant_1995}{Wegener
and Liebig 1995}) and cultural schemas
(\citeproc{ref-homan_being_2017}{Homan, Valentino, and Weed 2017}) is
pivotal in shaping these explanations by offering symbolic
representations that frame societal structures and expectations. While
significant attention has been paid to wage inequality, income
distribution, and payment differentials in the literature {[}Juan Carlos
Castillo (\citeproc{ref-castillo_legitimacy_2011}{2011}); {]} (Castillo,
2011; Evans et al., 2010; Jasso, 1999; Shariff et al., 2016),~ there has
been less examination of public beliefs about which life domains should
be governed by market relations and even less about children's
acceptance or rejection of these market principles. This oversight is
notable given the extensive encroachment of market logic into public
goods, welfare policy, and social services over the past five decades
(Centeno \& Cohen, 2012; Harvey, 2007), affecting areas such as
pensions, health services, and education. The justification of social
inequality based on marketype criteria has been conceptualized as the
individuals' adherence to one specific justice evaluation, the market
justice, that is, affording legitimacy to the allocation of goods and
services based on prices and individuals' ability to pay (Boltanski \&
Chiapello- 2007; Lane 1986; Streeck 2012).

Robert E. Lane proposed the underpinnings of market justice, which he
differentiated from political justice. For him, ``it is the genius of
the market to stimulate wants without at the same time stimulating a
sense of deserving more than one gets'' (1986, p.~384). Following the
theory of relative deprivation --a social phenomenon arising when
individuals cannot afford what most others in their environment can
(Merton 1950)-- Lane notes that, in market settings, social comparisons
are more likely to motivate increased effort rather than feelings of
acute injustice because individuals attribute outcomes to their actions.
Although empirical research has shown that, contrary to Lane's
observation, relative deprivation, even in market settings, produces
feelings of dissatisfaction, anger, and resentment that might motivate
forms of collective action such as protests and revolt (Greitemeyer and
Sagioglou 2016; Mishra and Carleton 2015; Smith et al.~2012; Power 2018;
Power, Madsen, and Morton 2020), the conceptualization of market justice
has been since then closely coupled to the merit principle for
allocating outcomes: the claim that the unequal levels of well-being
individuals enjoy ought to be, to some extent, a function of their
talents and efforts, regardless of their needs or membership, the two
latter being the realm of political justice and its closely coupled
principles of need --allocating outcomes to those who require them
most-- and equality --allocating the same outcome to everyone--, which
have been at the center of welfarism (see Wilson 2003). The underlying
notion of market justice also resonates in its application to welfare
regimes, providing a framework for understanding the varied global
approaches to managing social services.

The management of social services manifests in varying approaches across
nations, with substantial differences in funding and delivery methods
(Jensen, 2008; Stoy, 2014). Nordic countries, for example, predominantly
employ public agencies to produce and provide social services, funding
these through collective taxation and offering them in kind to the
majority of citizens. This system prioritizes political justice, placing
it above market mechanisms in accessing services. In contrast, other
countries rely more heavily on for-profit entities and private funding,
where service distribution depends mainly on individual financial
capacity of paying user fees, highlighting the influence of market
justice in service allocation. The trend toward marketization of welfare
services has been growing since the 1980s (Salamon, 1993), and this
shift is increasingly evident even in countries where market solutions
have traditionally had a minor role in social policy (Sivesind, 2017).

The question arises whether adults and children justify unequal access
to welfare services based on market justice principles. Influenced by
theories of policy feedback, which suggest that social welfare policies
can reinforce (positive feedback) or undermine (negative feedback)
previous policy trajectories
(\citeproc{ref-fernandez_positive_2013}{Fernandez and Jaime-Castillo
2013}) (Fernandez and Jaime-Castillo 2013; Pierson 2000; Weaver 2010),
citizens' beliefs about market justice are likely also shaped by the
institutional and social contexts they encounter. Indeed, the
justification of inequality in access to essential services like
education and health, based on one's ability to pay, shows significant
variation across countries, as demonstrated by international surveys,
although they do not usually include children in their samples. For
instance, the 2019 International Social Survey Program (ISSP) provides
insights into how adults perceive inequality in accessing welfare
services. Figure 1 illustrates the variation in agreement levels
regarding whether it is just for individuals with higher incomes to
purchase better education or healthcare. Notably, in 18 of the 29
countries analyzed, there is a greater justification for market
inequality in healthcare access than in education. Nevertheless, the
general sentiment typically ranges from `somewhat unjust' to 'neither
just nor unjust, with mixed feelings.

While contemporary policy developments have increasingly embraced the
marketization of social welfare in areas such as education, pensions,
and health services, questions about public acceptance of these changes
persist. Lindh (\citeproc{ref-lindh_public_2015}{2015}) analysis of ISSP
2009 data from 17 OECD countries reveals a general lack of support for
market-based distribution of social services, suggesting widespread
disapproval of market stratification of essential services. This finding
is corroborated by Soler-Martínez and colleagues' (2023) research from
Latinobarómetro 2020 across 18 Latin American countries, where concerns
about health and education access predominated over income inequality.
These results indicate that reforms toward welfare marketization are
typically driven by elite political decisions rather than grassroots
demand.~

Despite high-income inequality and limited social mobility in Latin
America, there is a prevalent belief that individuals are solely
responsible for their economic outcomes, a view that varies across the
region
(\citeproc{ref-buccaMeritBlameUnequal2016}{\textbf{buccaMeritBlameUnequal2016?}};
\citeproc{ref-chongMysteryDiscriminationLatin2008}{\textbf{chongMysteryDiscriminationLatin2008?}};
\citeproc{ref-torcheIntergenerationalMobilityInequality2014}{\textbf{torcheIntergenerationalMobilityInequality2014?}};
\citeproc{ref-salgadoInequalityStratificationLatin2023}{\textbf{salgadoInequalityStratificationLatin2023?}}).
The reliance on private welfare providers and widespread user fees
(Molyneux, 2008) adds complexity to this context. Yet, research on
children's justification of market-based inequalities in accessing
welfare services remains limited, especially in Latin America,
highlighting a significant gap in understanding how younger generations
view market-based access to welfare and whether these views are
associated with their meritocratic beliefs.~

\subsection{Meritocracy}\label{meritocracy}

The concept of meritocracy frequently appears nowadays when analyzing
cultural determinants of social inequalities. In general, it is
mentioned as a value associated with justice, as it would link efforts
and talents with rewards in an equitable manner. This normative sense is
quite far from its original formulation by Young
(\citeproc{ref-young_rise_1958}{1958}) in the satirical novel ``The Rise
of Meritocracy'', where it ironically represented a mechanism for
reproducing the inequalities of origin. The meritocratic ideal had
remained relatively unchallenged until a series of recent publications
turned into its potential consequences for maintaining social
inequality. Perhaps one of the most recent sources in this line is
Michael Sandel's ``The Tyranny of Merit'', where he strongly questions
the implications of carrying out the principle of merit in societies
that do not guarantee equal opportunities and that generate a feeling of
scarce recognition and appreciation of those who receive lesser rewards:
``In society's eyes, and perhaps also their own, their work no longer
signified a valued contribution to the common good.''
(\citeproc{ref-sandel_tyranny_2020}{Sandel 2020}, pp.).

Empirical research on meritocracy has increased along with the
philosophical-normative discussion on meritocracy in recent years.
Particularly from a sociological perspective, meritocracy has been used
in research on social mobility to characterize societies with low
mobility that threaten the meritocratic ideal
(\citeproc{ref-goldthorpe_myth_2003}{Goldthorpe 2003}). More recently,
sociology and social psychology research has attended to the subjective
aspects vis-a-vis beliefs in meritocracy. The label of beliefs in this
realm covers a series of areas, such as attitudes, perceptions, and
preferences (Castillo et al), whereby most of the link this subjective
dimensions to individual socio-structural factors and~ context-level
determinants. For instance, some studies have analyzed how those with
greater privileges believe more in meritocracy (Reynolds \& Chan 2014),
how greater economic inequality increases meritocratic beliefs
(\citeproc{ref-mijs_paradox_2021a}{\textbf{mijs\_paradox\_2021a?}}), and
how larger inequality affects meritocratic beliefs (Morris et al 2022).
Based on these findings, a research agenda has been reinforced on the
legitimizing role of meritocracy, in line with previous studies using
the concept of a just world (Lerner, Dalbert) and the theory of system
justification (Jost \& Major).

How do meritocratic beliefs legitimize inequalities? Empirical studies
have used experiments and surveys to address this question. For
instance, the evidence suggests that just world beliefs correlate
negatively with support for redistributive compensation systems
(\citeproc{ref-frank_performance_2015}{Frank, Wertenbroch, and Maddux
2015}). Conversely, individuals tend to support redistribution when they
believe that the disadvantaged lack the opportunities to succeed
(\citeproc{ref-evans_strong_2018}{Evans and Kelley 2018}). Almås,
Cappelen, and Tungodden (\citeproc{ref-almas_cutthroat_2020}{2020})
found that in a relatively unequal society (the United States), the
highly educated accept inequality significantly more than the less
educated because they perceive inequality as justifiable owing to
differences in productivity (i.e., merit), whereas in a relatively equal
society (Norway), the less educated accept inequality more, but not
significantly more than the highly educated because meritocratic values
are less prevalent. Barr and Miller
(\citeproc{ref-barr_effect_2020}{2020}) also addressed this triple
interaction between the level of inequality in a society, the individual
level of education, and the perceived origin of the disparity (either by
luck or effort) to determine the extent to which inequality is accepted.
They found that the interaction's mechanism varies depending on the
compared societies. Finally, García-Sánchez et al.
(\citeproc{ref-garcia-sanchez_attitudes_2020}{2020}), using data for 41
countries from the International Social Survey Programme (ISSP), found
that the perceived size of the income gap correlated positively with
support for progressive taxation. Still, this association was weaker
among those who endorsed meritocratic and equal opportunity beliefs. In
the same line, experimental research by Durante \& Putterman (2009)
subjects support less redistribution when the initial distribution is
determined according to task performance.

Research about meritocratic beliefs at school age is rather scarce,
leaving a wide research gap as schools are one of the primary
socialization institutions where achievement based on merit explains
success (Erivwo et al 2021). Nevertheless, it is possible to find some
initial works focused on the area of distributive justice at school that
are closely related to meritocratic beliefs, as the ones by Resh and
Sabbagh (quotes). Using justice in grade obtention as a measure of
distributive justice and meritocracy (Sabbagh et al 2006), they find for
instance that a larger sense of distributive justice about grades is
associated to higher socio-economic status
(\citeproc{ref-resh_sense_2010}{Resh 2010}), have a positive effect on
liberal democratic orientation and on trust in people and in formal
institutions (Resh \& Sabbagh 2013), and tend to refrain from violence
and to engage to a greater extent in extra-curricular school activity
and community volunteering (\citeproc{ref-resh_sense_2017}{Resh and
Sabbagh 2017}).

\subsection{Children's judgments of inequality, schools, and family
background}\label{childrens-judgments-of-inequality-schools-and-family-background}

Research indicates various socialization practices at families and
schools during childhood and adolescence that impact dispositional and
behavioral tendencies concerning justifications of inequality in adult
life. The differences in economic understanding across age groups are
consistent with research on cognitive development (Choudhury et al.,
2006). Adolescents with mature socio-cognitive abilities tend to express
stronger preferences for fairness than infants and children (Wynn et
al., 2017). As children grow older, they become more likely to behave
fairly, with their early-emerging strict egalitarianism being replaced
by an increasing endorsement of fairness principles and engagement in
collaborative activities (Huppert et al., 2019; McAuliffe et al., 2017).
In these activities, their fairness views consider individual
contributions, merits, and circumstances (Almås et al., 2010; Huppert et
al., 2019; Sigelman \& Waitzman, 1991). Engelmann and Tomasello (2019)
claim that children's sense of fairness emerges at three years old, and
we can observe it in collaborative activities, where they accept
inequality if the procedure gives everyone an equal chance. Therefore,
children at this age respond to unequal distributions based on
interpersonal concerns, as they already demand equal respect. In any
case, between 3 and 8 years of age, inequitable and anti-meritorious
allocations are evaluated more negatively, but equitable and meritorious
allocations are not evaluated more positively (Elenbaas, 2019).

Some research shows that the social environment in which children
develop, such as family and school, is associated with their prosocial
behaviors by playing an essential role in the transmission of equity
norms (Schunk \& Zipperle, 2023; Kosse et al., 2020). In fact, schools
contribute to institutionalizing and reproducing inequality by promoting
values, norms, practices, and languages familiar to higher-class
families because the dominant group's culture shapes educational
institutions (Bourdieu \& Passeron, 1990). Middle- and upper-class
students are better equipped to face academic challenges and are more
familiar with academic expectations (Mikus et al., 2019). Such
familiarity represents cultural capital in educational contexts because
higher-status students come to school ready to meet these expectations
and reap the benefits (Jack, 2016; Khan, 2011). Conversely, lower-status
children lacking cultural capital must catch up while experiencing
inequitable comparisons (Goudeau \& Croizet, 2017). Additionally,
academic achievement is treated as the outcome of dispositional factors
(e.g., pupils' efforts and talents or lack of them) rather than the
result of differential access to critical resources. Due to the
meritocratic frame schools encourage, both low- and high-status
individuals believe that students' success or failure is not due to
their family background but rather to differences in efforts and talents
(Darnon et al., 2018). In this sense, we believe that the perception of
meritocracy can influence students' judgments about market justice
preferences. Furthermore, we believe there is a difference between
students' perceptions of meritocracy based on their own experience in
school and what they perceive in society at large. Consistently, our
first two hypotheses are:

H1a: Students who perceive that there is more meritocracy at school will
show larger market justice preferences

H1b: Students who perceive that there is more meritocracy in society
will show larger market justice preferences

Family background and family socialization practices also contribute to
children's and adolescent's market justice preferences. For example,
Almås and colleagues (2017) found that adolescents from
low-socioeconomic-status families are likelier to have an egalitarian
fairness view and consider an equal distribution as fair in a situation
with unequal merits. The authors speculate that differences in
socialization practices across status groups might bring about, to a
great extent, the fairness views of children and adolescents because
social status seems to interact with these evaluations (e.g., Hvidberg
et al., 2023).~

The classic work of Kohn showed that middle-class parents value the
expression of internal states and emotions, such as self-control,
curiosity, happiness, and consideration, while working-class parents
promote deference, obedience, and conformity to authority (Kohn, 1963;
Kohn \& Schooler, 1969). Although parents from all social backgrounds
encourage individualism in their children, this shared norm translates
into different forms in high and low social classes (1999). Acemoglu
(2021) claimed that the values families impart to their children
interact with social mobility. Because obedience is a valuable
characteristic for employers, in low-wage and social mobility
environments, low-income families impart values of obedience to their
children to prevent disadvantaging them in labor markets. On the one
hand, children from privileged families are socialized to adopt a clear
conception of individualism that highlights their internal states,
independence, and idiosyncrasies. In contrast, children from
disadvantaged families are socialized to support a more balanced view of
individualism that considers personal characteristics as resources to
overcome collective impediments on the path to upward mobility
(Iacoviello \& Lorenzi-Cioldi, 2019). In this way, we believe that there
are differences in the socialization of values according to
socioeconomic differences that could influence market justice
preferences and, therefore, we propose the following hypothesis:

H2: Students from families of higher social status will show larger
market justice preferences.

Recent empirical research has demonstrated that the institutional design
of schools, coupled with the meritocratic ideology it fosters,
significantly influences children's and adolescents' views on inequality
and deservingness. For example, Jonsson and Beach's (2015) study
revealed that higher-status adolescents in Sweden tend to perpetuate
social class stereotypes while describing the vocational and academic
tracks. Academic track students are depicted as wealthy, intelligent,
ambitious, and diligent, while vocational track students are
characterized as poor, unambitious, unintelligent, and lackadaisical.
These stereotypes help individuals maintain a sense of superiority over
others and legitimize the prevailing social hierarchies and economic
disparities (Jost \& Burgess, 2000).

H3: Students from schools of higher social status will show larger
market justice preferences.

H4: Students from schools with higher average levels of academic
achievement will show larger market justice preferences.

{[}interacciones{]}

H5: The perception of meritocracy in school and society will moderate
the effect of family social status on market justice preferences.

H6: The perception of meritocracy in school and society will moderate
the effect of school status on market justice preferences.

H7: The perception of meritocracy in school and society will moderate
the effect of school academic achievement on market justice preferences.

\subsubsection{Summary of hypotheses}\label{summary-of-hypotheses}

\subsection{}\label{section}

\subsection{Methods}\label{methods}

\subsubsection{Data}\label{data}

The main data source for the analysis is the First Study of Citizenship
Education in Chile, carried out by the Education Quality Agency of the
Ministry of Education. The application date was November 9, 2017. The
target population of this study is Eighth-grade students from 242
schools. In the data, there are 8,589 students and 6,770 parents. This
database was analyzed with the R package ``ResponsePatterns'' to detect
possible repetitive and ``careless'' response patterns and thus
contribute to a higher quality of research data (Gottfried, et al 2022).
Responses from 171 students and 79 parents were removed, which, when
merged, gave a total of 6,270 valid cases.

The analysis of school variables includes data from the Ministry of
Education's SIMCE 2017 database. This database contains information at
the school level, such as the administrative dependency, its
socioeconomic classification, and the achievement scores obtained in the
mathematics and language census tests. It is available for free use on
the MINEDUC {[}web page{]}.

After eliminating missing cases, the final sample used in the analysis
was based on 5,047 students and parents of 231 schools for the dependent
variable of access to social services.

\subsubsection{Variables}\label{variables}

\textbf{Dependent variables}

This study has three dependent variables related to the justification of
social inequality in specific policy domains. The first asks whether
access to social services should be conditional on income, i,.e., ``It
is just that in Chile people who can pay have a better education for
their children''. Students rated their preferences using the following
responses: ``strongly disagree'', ``Disagree'', ``Agree'', and
''strongly agree''. An average index is built with these items
(Cronbach's Alpha = 0,86). (Apéndice: items en español y su
correspondiente traducción al inglés)

Table 1 shows the items used, their response categories, and their
frequencies.

\textbf{Independent variables}

For the primary independent variable, the perception of meritocracy,
five items address the perception of rewards according to effort and
intelligence at the school and societal levels. At the school level,
students answer whether ``Intelligence is important to get good grades''
and ``Effort is important to get good grades''. At the societal level,
students respond to the following questions: ``In Chile, people are
rewarded for their efforts'', ``In Chile, people get what they
deserve'', and ``In Chile, people are rewarded for their intelligence
and skills''. Each item was answered on a four-point scale ranging from
``Strongly disagree'' to ``Strongly agree''.

The rest of the independent variables are divided into individual and
school levels. At the individual level, family socioeconomic status was
measured by the parents' highest educational level and the number of
books at home. Likewise, an index of access to technology includes the
number of computers, tablets, and cell phones at home, as well as
whether there is an Internet connection. Table 2 shows the items used,
their response categories, and their frequency.

The school-level variables are the administrative dependency of the
school, the socioeconomic classification made by the Ministry of
Education, the level of performance in the SIMCE test of the school, and
the proportion of parents with university or postgraduate degrees. Table
3 shows the items used, response categories, and frequency.

\subsubsection{Methods}\label{methods-1}

The data has a hierarchical structure of students nested in schools, so
the model estimation is performed in a multilevel (random effects)
framework. This modeling approach lets us correctly estimate individual
and contextual effects in a single model. We estimate cumulative link
mixed models for the ordinal dependent variables, whereas we use linear
mixed effects models for the average index of inequality justification.

The hypotheses of this research were pre-registered in the Open Science
Framework platform of the Center for Open ScorrectlyOSF), the access to
the document is available at this
{[}link{]}(https://doi.org/10.17605/OSF.IO/UFSDV). The statistical
analysis of this research was performed using the free software R
version 4.1.3.

\subsection{Analysis}\label{analysis}

\subsubsection{Descriptive analysis}\label{descriptive-analysis}

Figure 3 shows a series of graphs depicting the association between the
variables of market justice preferences - in education, health, and
pensions - and the variables of meritocratic perception at school
(effort and talent) and in society (effort, talent, and deservingness)
(see conceptual diagram en figure X). On the left, we observe the social
meritocracy diagrams, while on the right the school meritocracy diagrams
are shown. For the three variables of perception of meritocracy in
society the relationship is clear, since the average of market justice
preferences increases the more there is agreement that people are
rewarded for their effort, merit, and talent. This relationship needs to
be clarified in the case of the variables of perception of meritocracy
at school. To the extent that there is more agreement that the
perception of talent is essential for obtaining good grades, the average
of market justice preferences increases, but this relationship is not as
clear as with the variables of meritocracy in society. In addition, the
graph does not show a clear trend in the relationship between the
perception that effort is essential to obtain good grades and market
justice preferences.

\subsubsection{Cumulative link mixed models for the Justification of
inequality in education, health and
pensions}\label{cumulative-link-mixed-models-for-the-justification-of-inequality-in-education-health-and-pensions}

Three Cumulative link mixed models were estimated for the ordinal
dependent variables of justice in differential access to pensions,
education, and health according to income. Figure 4 shows the estimation
of this regression model containing all the variables used in the study
for the three dependent variables separately. However, this figure shows
only the effect of meritocracy variables on society and school as
independent variables. Complete models are available in the appendix.

Figure 4 shows that for the three dependent variables of justice in
differential access to pensions, education, and health, the trend is
similar. In the context of school meritocracy, the effects are mixed. As
school talent increases, the justification for differentiated access to
pensions, education, and health increases; on the contrary, as the
school effort variable increases, the justification for differentiated
access to pensions, education, and health decreases, keeping the rest of
the independent variables constant. Regarding the variables of social
meritocracy, the three variables of talent, effort, and deservingness
show that as these increase, the justification for differentiated access
to pensions, education, and health also increases.

\subsubsection{Multilevel regression models for market justice
preferences}\label{multilevel-regression-models-for-market-justice-preferences}

Table 4 shows the results of the multilevel estimation for justice
market preferences. For this variable, the intraclass correlation
obtained shows that the variation between schools corresponds to 4\% of
the variation of students' preferences. This means that there is low
variance between schools and therefore limits the possibilities of
finding effects at the aggregate level.

Model 1 introduces social meritocratic variables: effort (whether
efforts are rewarded), deservingness (people get what they deserve), and
social talent (intelligence and skills are rewarded in society). In line
with our hypotheses, the perception of a meritocratic society is
positively related to the justification of inequality. Model 2 shows a
mixed picture: while those perceiving that talent is rewarded also
justify the inequality, the perception that effort is rewarded at school
is negatively related to justification of inequality. Family background
variables in Model 3 reveal that education and technology access are not
related to justification of inequality, whereby we observe a negative
impact of family cultural capital as measured by the number of books at
home. While school socio-structural variables added in Model 4 show no
significant effects, average achievement scores in the SIMCE test depict
a negative relationship with the dependent variable, meaning that
students that attend schools with better achievement scores on average
justify less inequality.~

In relation to model fit, when comparing the deviance with a model
without predictors (null model), all the models have a statistically
significant difference, with model 5 having the lowest deviance.
According to Raudenbush and Bryk's (2002) estimate of R2, the level 1
variance of model 5 is 0.11 and the level 2 variance is 0.72. The total
variance of model 5 according to Snijders and Bosker (2012) is 0.14.

\subsubsection{Interactions effects}\label{interactions-effects}

The interaction terms in Table 6 suggest that students' family
background moderates the relationship between their meritocratic
perceptions in Chile and their justification of inequality. The
direction of these effects confirms our initial prediction. Thus, in
line with our Hypothesis 3, the relationship between social effort and
justification of inequality becomes less positive for those students
whose parents achieved university or postgraduate education. That is,
lower-status students (measured by parental education) justify more
inequality when they adhere more strongly to deservingnessmeritocratic
perceptions in Chile. We depict this moderating effect in Figure 1. This
result confirms the enlightenment thesis (see above). The same
moderating effect is observed for students' family cultural capital: the
relationship between effort (but not deservingness nor social talent)
and justification of inequality becomes less positive for students with
more than 25 books at home. Interestingly, family cultural capital also
moderates the relationship between meritocratic beliefs at school (as
measured by students' belief that effort is important to get good
grades) and justification of inequality.

At the school level, we also observe moderating effects of school status
and students' meritocratic beliefs over their justification of
inequality. Thus, in line with our Hypothesis 5, high-status schools
justify less inequality when, on average, their students have a greater
perception of meritocracy in Chile, as measured by the three indicators
we used (i.e., effort, deservingness, and talent). Finally, as we stated
in Hypothesis 7, lowhigh-achieving schools justify more inequality when
their students have, on average, stronger meritocratic beliefs in Chile.
In Figure 25 we depict this moderating effect of school achievement for
the relationship between deservingness and justification of inequality.
We did not observe these moderating effects at the school level for
students' meritocratic beliefs in the school (i.e., the idea that effort
or talent are important to get good grades).

\subsection{Conclusion}\label{conclusion}

\begin{itemize}
\tightlist
\item
  Efecto ilustrador de la educación? (esfuerzo -\textgreater{}
  redistribución)
\end{itemize}

\subsection{Appendix}\label{appendix}

\begin{longtable}[]{@{}
  >{\raggedright\arraybackslash}p{(\columnwidth - 2\tabcolsep) * \real{0.4496}}
  >{\raggedright\arraybackslash}p{(\columnwidth - 2\tabcolsep) * \real{0.5504}}@{}}
\toprule\noalign{}
\endhead
\bottomrule\noalign{}
\endlastfoot
\textbf{English translation} & \textbf{Original Spanish} \\
It is just that in Chile people who cna pay have a better education for
their children & Es justo que en Chile las personas que puedan pagar
tengan una mejor educación para sus hijos. \\
It is just that in Chile people with higher incomes can have better
pensions than people with low incomes & Es justo que en Chile las
personas con mayores ingresos puedan tener mejores pensiones que las
personas de ingresos más bajos \\
It is just that in Chile people with higher incomes can access better
health services than people with low incomes & Es justo que en Chile las
personas con mayores ingresos puedan acceder a una mejor atención de
salud que las personas con ingresos más bajos \\
\end{longtable}

\phantomsection\label{refs}
\begin{CSLReferences}{1}{0}
\bibitem[\citeproctext]{ref-almas_cutthroat_2020}
Almås, Ingvild, Alexander W. Cappelen, and Bertil Tungodden. 2020.
{``Cutthroat Capitalism Versus Cuddly Socialism: Are Americans More
Meritocratic and Efficiency-Seeking Than Scandinavians?''} \emph{Journal
of Political Economy} 128 (5): 1753--88.
\url{https://doi.org/10.1086/705551}.

\bibitem[\citeproctext]{ref-barr_effect_2020}
Barr, Abigail, and Luis Miller. 2020. {``The Effect of Education, Income
Inequality and Merit on Inequality Acceptance.''} \emph{Journal of
Economic Psychology} 80 (May): 102276.
\url{https://doi.org/10.1016/j.joep.2020.102276}.

\bibitem[\citeproctext]{ref-batruch_belief_2022}
Batruch, Anatolia, Jolanda Jetten, Herman Van de Werfhorst, Céline
Darnon, and Fabrizio Butera. 2022. {``Belief in School Meritocracy and
the Legitimization of Social and Income Inequality.''} \emph{Social
Psychological and Personality Science}, August, 194855062211110.
\url{https://doi.org/10.1177/19485506221111017}.

\bibitem[\citeproctext]{ref-bourdieu_reproduction_1990}
Bourdieu, Pierre, and Jean Claude Passeron. 1990. \emph{Reproduction in
Education, Society and Culture}. Second Edition. Sage Publications Ltd.

\bibitem[\citeproctext]{ref-castillo_legitimacy_2011}
Castillo, Juan Carlos. 2011. {``Legitimacy of Inequality in a Highly
Unequal Context: Evidence from the Chilean Case.''} \emph{Social Justice
Research} 24 (4): 314--40.
\url{https://doi.org/10.1007/s11211-011-0144-5}.

\bibitem[\citeproctext]{ref-castillo_meritocracia_2019}
Castillo, Juan C., Alex Torres, Jorge Atria, and Luis Maldonado. 2019.
{``Meritocracia y Desigualdad Económica: Percepciones, Preferencias e
Implicancias.''} \emph{Revista Internacional de Sociología} 77 (1): 117.
\url{https://doi.org/10.3989/ris.2019.77.1.17.114}.

\bibitem[\citeproctext]{ref-chafel_schooling_1997}
Chafel, Judith A. 1997. {``Schooling, the Hidden Curriculum, and
Children's Conceptions of Poverty.''} \emph{Social Policy Report} 11
(1): 1--28. \url{https://doi.org/10.1002/j.2379-3988.1997.tb00004.x}.

\bibitem[\citeproctext]{ref-duru-bellat_who_2012}
Duru-Bellat, Marie, and Elise Tenret. 2012. {``Who's for Meritocracy?
Individual and Contextual Variations in the Faith.''} \emph{Comparative
Education Review} 56 (2): 223--47. \url{https://doi.org/10.1086/661290}.

\bibitem[\citeproctext]{ref-evans_strong_2018}
Evans, M. D. R., and Jonathan Kelley. 2018. {``Strong Welfare States Do
Not Intensify Public Support for Income Redistribution, but Even Reduce
It Among the Prosperous: A Multilevel Analysis of Public Opinion in 30
Countries.''} \emph{Societies} 8 (4): 105.
\url{https://doi.org/10.3390/soc8040105}.

\bibitem[\citeproctext]{ref-fernandez_positive_2013}
Fernandez, J. J., and A. M. Jaime-Castillo. 2013. {``Positive or
Negative Policy Feedbacks? Explaining Popular Attitudes Towards
Pragmatic Pension Policy Reforms.''} \emph{European Sociological Review}
29 (4): 803--15. \url{https://doi.org/10.1093/esr/jcs059}.

\bibitem[\citeproctext]{ref-frank_performance_2015}
Frank, Douglas H., Klaus Wertenbroch, and William W. Maddux. 2015.
{``Performance Pay or Redistribution? Cultural Differences in Just-World
Beliefs and Preferences for Wage Inequality.''} \emph{Organizational
Behavior and Human Decision Processes} 130 (September): 160--70.
\url{https://doi.org/10.1016/j.obhdp.2015.04.001}.

\bibitem[\citeproctext]{ref-garcia-sanchez_attitudes_2020}
García-Sánchez, Efraín, Danny Osborne, Guillermo B. Willis, and Rosa
Rodríguez-Bailón. 2020. {``Attitudes Towards Redistribution and the
Interplay Between Perceptions and Beliefs about Inequality.''}
\emph{British Journal of Social Psychology} 59 (1): 111--36.
\url{https://doi.org/10.1111/bjso.12326}.

\bibitem[\citeproctext]{ref-goldthorpe_myth_2003}
Goldthorpe, John. 2003. {``The Myth of Education-Based Meritocracy.''}
\emph{New Economy} 10 (4): 234--39.
\url{https://doi.org/10.1046/j.1468-0041.2003.00324.x}.

\bibitem[\citeproctext]{ref-homan_being_2017}
Homan, Patricia, Lauren Valentino, and Emi Weed. 2017. {``Being and
Becoming Poor: How Cultural Schemas Shape Beliefs about Poverty.''}
\emph{Social Forces} 95 (3): 1023--48.
\url{https://doi.org/10.1093/sf/sox007}.

\bibitem[\citeproctext]{ref-kluegel_beliefs_1987}
Kluegel, James R., and Eliot R. Smith. 1987. \emph{Beliefs about
Inequality: Americans' Views of What Is and What Ought to Be}. New York:
Routledge.

\bibitem[\citeproctext]{ref-lampert_meritocratic_2013}
Lampert, Khen. 2013. \emph{Meritocratic Education and Social
Worthlessness}. London: Palgrave Macmillan UK.
\url{https://doi.org/10.1057/9781137324894}.

\bibitem[\citeproctext]{ref-lindh_public_2015}
Lindh, Arvid. 2015. {``Public Opinion Against Markets? Attitudes Towards
Market Distribution of Social Services -- a Comparison of 17
Countries.''} \emph{Social Policy \& Administration} 49 (7): 887--910.
\url{https://doi.org/10.1111/spol.12105}.

\bibitem[\citeproctext]{ref-mijs_stratified_2016}
Mijs, Jonathan. 2016. {``Stratified Failure: Educational Stratification
and Students' Attributions of Their Mathematics Performance in 24
Countries.''} \emph{Sociology of Education} 89 (2): 137--53.
\url{https://doi.org/10.1177/0038040716636434}.

\bibitem[\citeproctext]{ref-mijs_paradox_2021}
Mijs, Jonathan J B. 2021. {``The Paradox of Inequality: Income
Inequality and Belief in Meritocracy Go Hand in Hand.''}
\emph{Socio-Economic Review} 19 (1): 7--35.
\url{https://doi.org/10.1093/ser/mwy051}.

\bibitem[\citeproctext]{ref-resh_sense_2010}
Resh, Nura. 2010. {``Sense of Justice about Grades in School: Is It
Stratified Like Academic Achievement?''} \emph{Social Psychology of
Education} 13 (3): 313--29.
\url{https://doi.org/10.1007/s11218-010-9117-z}.

\bibitem[\citeproctext]{ref-resh_sense_2017}
Resh, Nura, and Clara Sabbagh. 2017. {``Sense of Justice in School and
Civic Behavior.''} \emph{Social Psychology of Education} 20 (2):
387--409. \url{https://doi.org/10.1007/s11218-017-9375-0}.

\bibitem[\citeproctext]{ref-sandel_tyranny_2020}
Sandel, Michael J. 2020. \emph{The Tyranny of Merit: What's Become of
the Common Good?} First edition. New York: {Farrar, Straus and Giroux}.

\bibitem[\citeproctext]{ref-vonhippel_test_2019}
Von Hippel, Paul, and Caitlin Hamrock. 2019. {``Do Test Score Gaps Grow
Before, During, or Between the School Years? Measurement Artifacts and
What We Can Know in Spite of Them.''} \emph{Sociological Science} 6:
43--80. \url{https://doi.org/10.15195/v6.a3}.

\bibitem[\citeproctext]{ref-wegener_dominant_1995}
Wegener, Bernd, and Stefan Liebig. 1995. {``Dominant Ideologies and the
Variation of Distributive Justice Norms: A Comparison of East and West
Germany, and the United States.''} In \emph{Social Justice and Political
Change: Public Opinion in Capitalist and Post-Communist States}, edited
by James R. Kluegel, David S. Mason, and Bernd Wegener, 239--59. New
York: Walter de Gruyter.

\bibitem[\citeproctext]{ref-wiederkehr_belief_2015}
Wiederkehr, Virginie, Virginie Bonnot, Silvia Krauth-Gruber, and Céline
Darnon. 2015. {``Belief in School Meritocracy as a System-Justifying
Tool for Low Status Students.''} \emph{Frontiers in Psychology} 6.
\url{https://www.frontiersin.org/articles/10.3389/fpsyg.2015.01053}.

\bibitem[\citeproctext]{ref-young_rise_1958}
Young, Michael. 1958. \emph{The Rise of the Meritocracy}. New Brunswick,
N.J., U.S.A: Transaction Publishers.

\end{CSLReferences}



\end{document}
